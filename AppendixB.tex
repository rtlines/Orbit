\chapter{Including Your Code}
\label{app:Talk}

You might want to include some code in your thesis. This is done with a verbatim environment so \LaTeX \enspace doesn't try to interpret the code as \LaTeX \enspace commands. I want to make sure the code fits on the page, so I am going to reduce the font size with a small environment as well. 
\begin{small}

	
	\begin{verbatim}
		##############################################################
		# second order Runga Cutta for a projectile with expansion   #
		##############################################################
		# Todd Lines
		# 2020 08 14
		##############################################################
		# 
		##############################################################
		# Load Libraries
		import numpy as np
		#import matplotlib
		#matplotlib.use('tkagg')
		import matplotlib.pyplot as plt
		
		##############################################################
		# Initial conditions and physial setup constants
		R=6371E3      # radius of Earth in m
		M=5.9E24       # Mass of plannet in Kg
		m=0.020       # mass of sattelite in kilograms
		G=5.11E-11    # N*m**2/kg**2
		x0=2*R          # initial x position in m
		y0=0.5*R       # initial y position in m
		v0=6000.0         # initial velocity in m/s
		thetadeg=100   # launch angle in degrees
		Rexp=0.007      # expansion coefficient in m/s/m This is really 
		# a type of Hubble constant, but I am not using
		# the right value yet.
		##############################################################
		## Set up the time steps and number of calcualtions
		deltat=1.0        # Time steps of 0.01 seconds
		ti=0              # starting at t=0
		tf=200000          # final time
		N=int((tf-ti)/deltat)  # calcualte how many time steps are in 20 seconds
		##############################################################
		# Preliminary calculations
		pi=np.arccos(-1.0)           # calculate pi to machine percision
		theta=thetadeg*pi/180   # calcualte theta in radians
		vx1=v0*np.cos(theta)       # calculte the x component of the initial velocity
		vy1=v0*np.sin(theta)       # calculte the y component of the initial velocity
		
		##############################################################
		# define and zero arrays
		t=np.zeros((N))
		x=np.zeros((N))
		y=np.zeros((N))
		z=np.zeros((N))
		vx=np.zeros((N))
		vy=np.zeros((N))
		xnd=np.zeros((N))
		ynd=np.zeros((N))
		xE=np.zeros((N))
		yE=np.zeros((N))
		
		
		##############################################################
		## make an array of times to use
		t=np.linspace(0,tf,num=N);
		##############################################################
		# Draw The Earth
		r=R      # distance from center in m
		ThetaE0=0.00      # initial angle in degrees
		delta_ThetaE=5  # change in angle in degrees
		Ne= int(360/delta_ThetaE)+1       # number of points
		ye=np.zeros((Ne))
		xe=np.zeros((Ne))
		##############################################################
		# Draw Circle
		ye[0]=0        # initial y positoin
		xe[0]=r
		thetaE=ThetaE0
		for i in range (0,Ne):
		thetaE=thetaE+delta_ThetaE
		xe[i]=r*np.cos(thetaE*pi/180)
		ye[i]=r*np.sin(thetaE*pi/180)
		plt.plot(xe,ye)
		##############################################################
		# now perform an Euler's Method Calculation
		# now recalling that vx(i) already has a cos(theta) in it,
		# we can use this to calculate the x part of the resistive
		# force and likewise use vy(i) in calculating the y part of
		# the resistive force. No explicit calculation of theta is
		# necessary this way, and we save lots of computation time.
		x=np.zeros((N))
		y=np.zeros((N))
		vx=np.zeros((N))
		vy=np.zeros((N))
		x[0]=x0                        # initial x position
		y[0]=y0                        # initial y positoin
		vx[0]=vx1                      # initial x velocity
		vy[0]=vy1            # initial y velocity, what we give it
		
		# plus the expansion sudo velocity
		print('working')
		for i in range (0,N-1):
		r=np.sqrt(x[i]**2+y[i]**2)
		if x[i] != 0:
		theta=np.arctan(y[i]/x[i])
		else:
		theta=pi/2
		
		if y[i]>0 and x[i]<0:
		theta=pi+theta
		if y[i]<0 and x[i]<=0:
		theta=theta+pi
		if y[i]<0 and x[i]>0:
		theta=2*pi+theta
		if r<R:
		print("hit surface")
		break   #hit surface
		
		fx=vx[i]
		gx=-np.cos(theta) * G*M/r**2
		
		
		fy=vy[i]
		gy=-np.sin(theta) * G*M/r**2
		
		x[i+1]=x[i]+deltat*fx
		y[i+1]=y[i]+deltat*fy
		vx[i+1]=vx[i]+deltat*gx
		vy[i+1]=vy[i]+deltat*gy    # once again I have added in the 
		# expansion term
		#print(i, theta, x[i],y[i],vx[i],vy[i], gx, gy)
		
		print('done')
		plt.plot(x,y)
		#xEu=np.copy(x)
		#yEu=np.copy(y)
		#plt.plot(xEu,yEu, label='expansion2')
		
		
		###############################################################
		##now perform an RK2 Method Calculation
		#x[0]=x0                    # initial x position
		#y[0]=y0                    # initial y positoin
		#vx[0]=vx1                  # inintal x velocity
		#vy[0]=vy1+Rexp*y[0]        # initial y velocity, what we give it
		#                           # plus the expansion sudo velocity
		#for i in range (0,N-1):
		#    v=np.sqrt(vx[i]**2+vy[i]**2)
		#    # first the Euler step, This is tricky because I want just the x
		#    # component and the y component of the drag force, but the drag force
		#    # depends on v^2. Remembering that vx=v*cos(theta), we can then multiply
		#    # the speed, v, by vx to get v^2*cos(theta). This way we don't have to
		#    # calculate theta explicitly
		#    Rx=0.5*D*rho*pi*r*r*v*vx[i]
		#    Ry=0.5*D*rho*pi*r*r*v*vy[i]
		#    fx=vx[i]
		#    gx=-Rx/m
		#    fy=vy[i]
		#    gy=-g-Ry/m
		#    kx1=deltat*fx
		#    ky1=deltat*fy
		#    kvx1=deltat*gx
		#    kvy1=deltat*gy
		#    #now the RK step, What to do with the v^2? I think we can do the same
		#    #thing as above to find the x abd y components of the velocity.
		#    v=np.sqrt((vx[i]+kvx1/2)**2+(vy[i]+kvy1/2)**2)
		#    Rx2=0.5*D*rho*pi*r*r*v*(vx[i]+kvx1/2)
		#    Ry2=0.5*D*rho*pi*r*r*v*(vy[i]+kvy1/2)
		#    fx2=vx[i]+kvx1/2
		#    gx2=-Rx/m
		#    fy2=vy[i]+kvy1/2
		#    gy2=-g-Ry/m
		#    #finally take the RK step.
		#    x[i+1]=x[i]+deltat*fx2
		#    y[i+1]=y[i]+deltat*fy2
		#    vx[i+1]=vx[i]+deltat*gx2
		#    vy[i+1]=vy[i]+deltat*gy2+Rexp*y[i]     # once again I have added in the 
		#                                            # expansion term
		#    #if y(i+1)<=0.0 , break, end
		#    
		#xRK=np.copy(x)
		#yRK=np.copy(y)
		#
		#plt.plot(xnd,ynd, label='kinematic')
		#plt.plot(xEu,yEu, label='expansion')
		#plt.plot(xRK,yRK,label='RK')
		#plt.xlabel('x [m]')
		#plt.ylabel('y [m]')
		#leg=plt.legend();
		plt.show()

	
	\end{verbatim}
	
\end{small}


