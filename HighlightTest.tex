\documentclass[12pt]{article}
\usepackage[english]{babel}
\usepackage[utf8x]{inputenc}
\usepackage{fullpage}

\usepackage{listings}
\usepackage{color}

\title{Syntax Highlighting in LaTeX with the listings Package and more}
\author{writeLaTeX}

%\definecolor{mygreen}{rgb}{0,0.6,0}
%\definecolor{mygray}{rgb}{0.5,0.5,0.5}
%\definecolor{mymauve}{rgb}{0.58,0,0.82}
%%%%%%%%%%%%%%%%%%%%%%%%%%%%%%%%%%%%%%%%%%%%%%%%%%%%%%%%%%%%%%%%%%%%%%%%%%%%%%%% 
%%% ~ Arduino Language - Arduino IDE Colors ~                                  %%%
%%%                                                                            %%%
%%% YidongQIN | 06/20/2019 | No Licence given                                  %%%
%%%  https://gist.github.com/YidongQIN/a10dd4f72381362aff4257e7a5541d86        %%%
%%% -------------------------------------------------------------------------- %%%
%%%                                                                            %%%
%%% Place this file in your working directory (next to the latex file you're   %%%
%%% working on).  To add it to your project, place:                            %%%
%%%    \input{pythonCodeHighLighting.tex}                                             %%%
%%% somewhere before \begin{document} in your latex file.                      %%%
%%%                                                                            %%%
%%% In your document, place your arduino code between:                         %%%
%%%   \begin{lstlisting}[language=Arduino]                                     %%%
%%% and:                                                                       %%%
%%%   \end{lstlisting}                                                         %%%
%%%                                                                            %%%
%%% Or create your own style to add non-built-in functions and variables.      %%%
%%%                                                                            %%%
%%%%%%%%%%%%%%%%%%%%%%%%%%%%%%%%%%%%%%%%%%%%%%%%%%%%%%%%%%%%%%%%%%%%%%%%%%%%%%%% 
% color def
\usepackage{color}
\definecolor{darkred}{rgb}{0.6,0.0,0.0}
\definecolor{darkgreen}{rgb}{0,0.50,0}
\definecolor{lightblue}{rgb}{0.0,0.42,0.91}
\definecolor{orange}{rgb}{0.99,0.48,0.13}
\definecolor{grass}{rgb}{0.18,0.80,0.18}
\definecolor{pink}{rgb}{0.97,0.15,0.45}

% listings
%\usepackage{listings}

% General Setting of listings
\lstset{
	aboveskip=1em,
	breaklines=true,
	abovecaptionskip=-6pt,
	captionpos=b,
	escapeinside={\%*}{*)},
	frame=single,
	numbers=left,
	numbersep=15pt,
	numberstyle=\tiny,
}
% 0. Basic Color Theme
\lstdefinestyle{colored}{ %
	basicstyle=\ttfamily,
	backgroundcolor=\color{white},
	commentstyle=\color{green}\itshape,
	keywordstyle=\color{blue}\bfseries\itshape,
	stringstyle=\color{red},
}
% 1. General Python Keywords List
\lstdefinelanguage{PythonPlus}[]{Python}{
	morekeywords=[1]{,as,assert,nonlocal,with,yield,self,True,False,None,} % Python builtin
	morekeywords=[2]{,__init__,__add__,__mul__,__div__,__sub__,__call__,__getitem__,__setitem__,__eq__,__ne__,__nonzero__,__rmul__,__radd__,__repr__,__str__,__get__,__truediv__,__pow__,__name__,__future__,__all__,}, % magic methods
	morekeywords=[3]{,object,type,isinstance,copy,deepcopy,zip,enumerate,reversed,list,set,len,dict,tuple,range,xrange,append,execfile,real,imag,reduce,str,repr,}, % common functions
	morekeywords=[4]{,Exception,NameError,IndexError,SyntaxError,TypeError,ValueError,OverflowError,ZeroDivisionError,}, % errors
	morekeywords=[5]{,ode,fsolve,sqrt,exp,sin,cos,arctan,arctan2,arccos,pi, array,norm,solve,dot,arange,isscalar,max,sum,flatten,shape,reshape,find,any,all,abs,plot,linspace,legend,quad,polyval,polyfit,hstack,concatenate,vstack,column_stack,empty,zeros,ones,rand,vander,grid,pcolor,eig,eigs,eigvals,svd,qr,tan,det,logspace,roll,min,mean,cumsum,cumprod,diff,vectorize,lstsq,cla,eye,xlabel,ylabel,squeeze,}, % numpy / math
}
% 2. New Language based on Python
\lstdefinelanguage{PyBrIM}[]{PythonPlus}{
	emph={d,E,a,Fc28,Fy,Fu,D,des,supplier,Material,Rectangle,PyElmt},
}
% 3. Extended theme
\lstdefinestyle{colorEX}{
	basicstyle=\ttfamily,
	backgroundcolor=\color{white},
	commentstyle=\color{darkgreen}\slshape,
	keywordstyle=\color{blue}\bfseries\itshape,
	keywordstyle=[2]\color{blue}\bfseries,
	keywordstyle=[3]\color{grass},
	keywordstyle=[4]\color{red},
	keywordstyle=[5]\color{orange},
	stringstyle=\color{darkred},
	emphstyle=\color{pink}\underbar,
}

%\lstset{ %
%	backgroundcolor=\color{white},   % choose the background color
%	basicstyle=\footnotesize,        % size of fonts used for the code
%	breaklines=true,                 % automatic line breaking only at whitespace
%	captionpos=b,                    % sets the caption-position to bottom
%	commentstyle=\color{mygreen},    % comment style
%	escapeinside={\%*}{*)},          % if you want to add LaTeX within your code
%	keywordstyle=\color{blue},       % keyword style
%	stringstyle=\color{mymauve},     % string literal style
%}

\begin{document}
	
	\maketitle
	

	
	\section{Python}
	% from http://wiki.scipy.org/Numpy_Example_List
	\begin{lstlisting}[language=Python]
		>>> from numpy import *
		>>> from numpy.fft import *
		>>> signal = array([-2., 8., -6., 4., 1., 0., 3., 5.])
		>>> fourier = fft(signal)
		>>> N = len(signal)
		>>> timestep = 0.1 # if unit=day -> freq unit=cycles/day
		>>> freq = fftfreq(N, d=timestep) # freqs corresponding to 'fourier'
		>>> freq
		array([ 0. , 1.25, 2.5 , 3.75, -5. , -3.75, -2.5 , -1.25])
		>>> fftshift(freq) # freqs in ascending order
		array([-5. , -3.75, -2.5 , -1.25, 0. , 1.25, 2.5 , 3.75])
	\end{lstlisting}
	

	
\end{document}