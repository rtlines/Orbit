\documentclass[12pt]{article}
\usepackage[english]{babel}
\usepackage[utf8x]{inputenc}
\usepackage{fullpage}

\usepackage{listings}
\usepackage{color}

\title{Syntax Highlighting in LaTeX with the listings Package and more}
\author{writeLaTeX}

%\definecolor{mygreen}{rgb}{0,0.6,0}
%\definecolor{mygray}{rgb}{0.5,0.5,0.5}
%\definecolor{mymauve}{rgb}{0.58,0,0.82}
\input{python.tex}

%\lstset{ %
%	backgroundcolor=\color{white},   % choose the background color
%	basicstyle=\footnotesize,        % size of fonts used for the code
%	breaklines=true,                 % automatic line breaking only at whitespace
%	captionpos=b,                    % sets the caption-position to bottom
%	commentstyle=\color{mygreen},    % comment style
%	escapeinside={\%*}{*)},          % if you want to add LaTeX within your code
%	keywordstyle=\color{blue},       % keyword style
%	stringstyle=\color{mymauve},     % string literal style
%}

\begin{document}
	
	\maketitle
	

	
	\section{Python}
	% from http://wiki.scipy.org/Numpy_Example_List
	\begin{lstlisting}[language=Python]
		>>> from numpy import *
		>>> from numpy.fft import *
		>>> signal = array([-2., 8., -6., 4., 1., 0., 3., 5.])
		>>> fourier = fft(signal)
		>>> N = len(signal)
		>>> timestep = 0.1 # if unit=day -> freq unit=cycles/day
		>>> freq = fftfreq(N, d=timestep) # freqs corresponding to 'fourier'
		>>> freq
		array([ 0. , 1.25, 2.5 , 3.75, -5. , -3.75, -2.5 , -1.25])
		>>> fftshift(freq) # freqs in ascending order
		array([-5. , -3.75, -2.5 , -1.25, 0. , 1.25, 2.5 , 3.75])
	\end{lstlisting}
	

	
\end{document}