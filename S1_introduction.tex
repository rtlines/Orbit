\chapter{Overview}


\paragraph{``Silencer" and ``suppressor" are descriptive words for a device that acts as a sound moderator, like a muffler of a vehicle which would lessen exposure to damaging sound pressure levels. For the entirety of this paper I will refer to devices whose purpose is to lessen sound volume as ``suppressors." Unfortunately, with this general description of suppressors, most suppressors invented so far tend to focus solely on shooter hearing protection by noise suppression in decibels. Other common points of optimization are suppressor longevity or heat management. These two variables, when optimized, are sufficient for a bolt-action style riffle. The following section reviews additional variables that could also be optimization to further protect the shooter, particularly for automatic and semiautomatic fire.\cite{bull2004}}





\section{Background}

\paragraph{For firearms more advanced than the bolt action rifle, more variables need to be considered than just sound pressure levels. The variables of importance are bolt velocity, blow-back, sound pressure level and spectral profile, spectral profile being a previously ignored variable of importance. Bolt velocity has the potential with enough force to break the upper as it overpowers the return spring. Blow-back is the spent gasses from the discharged projectile, and can cause negative lead exposure, impacting smell, sight and taste. All of these can cause health concerns.}






\subsection{Semiautomatic AR15 Platform}

\paragraph{The AR15 is a more advanced rifle so it needs a more advanced suppression solution than a bolt action rifle. The design of the AR15 that is common today started with a patent by Eugene M Stoner. The patent is for the direct impingement system. This system ultimately pushes the bolt back and allows the next round to be cycled in and the previously fired round to be ejected out of the barrel.\cite{waybackit2010}}







\paragraph{The AR15 can be called a partial impingement system. The difference is that the cup on the bolt has a redirection tube that moves the spent gasses into the bolt itself and out three relief holes. In this way, the direct impingement design is modified to create the new AR15 style rifles.}

\begin{figure}[!h]
	\centering
	\includegraphics[scale=.2]{AR10_bolt.png}
	\caption[Bolt design]{There are three important features of this AR bolt. The forward most part is the location that the firing pin comes to ignite the bullet cartridge. The second is the tube at the top of the bolt that receives the recycled gasses from the barrel when the shot is fired. Last, the three relief holes on the side of the bolt; these are the gas relief holes to not force the bolt back too powerfully.\cite{commonswikimedia}}
	\label{fig:AR bolt}
\end{figure}

\paragraph{A further description of the AR15 process is from Dr. Ryan Nielson of Brigham Young University-Idaho,}

\begin{quote}
	
	``Many of the problems associated with using a suppressor are meaningful in the context of how the firearms operate.  A description of the cycle of operation follows for the AR platform.
	
	
	
	Auto loading rifles need to keep the cartridge locked in the chamber until the gas pressure has dropped to safe levels.  This locking is accomplished by lugs on the bolt which turn to engage matching lugs in the rear of the barrel.
	
	
	
	Firing a cartridge causes energy from combustion gasses to be tapped off from the barrel with the final purpose of unlocking the bolt after pressure has dropped, and extracting the spent cartridge.
	
	
	
	The gasses are generally tapped from the barrel near muzzle, usually10-15" from the chamber, and carried down a gas tube into the upper-receiver of the rifle.  In the upper receiver, the gas tube pokes into the end of a bent tube attached to the bolt carrier, called the gas key.  The gas key channels gasses into the body of the bolt carrier, behind the bolt.
	
	
	
	The bolt rides in the carrier like a piston.  The gasses enter the “chamber” (bolt carrier), forcing the carrier rearward.  The carrier has an angled slot for a pin which connects to the bolt.  As the bolt carrier, like a moving cylinder of a piston, moves backwards, the pin engages the angled slot, turning the bolt and unlocking it so a cartridge case can be extracted.   Excess gasses from this chamber can be seen to exhaust through holes in the bolt carrier body in the left of the figure \ref{fig:Avg suppressor blow-back}
	
	
	
	Too much pressure, too quickly can open the bolt too soon, giving the carrier too much speed.  It can attempt to open the bolt early enough that the cartridge case is still held in place by the chamber pressure.  This can result in tearing the rim from a case, leaving it jammed in the chamber.   This is likely one of the failures which gave the M-16 a bad name in Vietnam.
	
	
	
	This can also result in a ruptured cartridge case, which certainly would disable the rifle, and possibly injure the shooter.
	
	
	
	Even the most moderate high-pressure events can batter the bolt into the buffer system, weakening or damaging it, as well as venting a lot of gas out the holes in the bolt carrier group, or past the extracting cartridge case."
	
\end{quote}






\subsection{Suppressors, Forwarding Devices and Muzzle Brakes}

\paragraph{Many devices used for the muzzle of a rifle serve various purposes of safety and conveniences for the operator. A muzzle brake is used to redirect the spent gasses, leaving the muzzle to keep the barrel in the same place and not rise from the recoil force of firing the round. Muzzle brakes come in many forms- some tunable and some for specific loads of ammunition. Forwarding devices are similar but unlike muzzle brakes. Forwarding devices move the spent gasses and muzzle blast away from the shooter prioritizing the comfort of the operator, over fast firing accuracy. As mentioned in the overview, the suppressor, which is the basis of this work, largely has a focus of reducing the sound pressure levels to the shooter's ears, plus the positive byproduct of these devices of recoil reduction.\cite{recoilweb2022}\cite{opticsplanet2021}\cite{nraila2022}}





\subsection{Bolt Velocity and Blow-back}
\paragraph{While using suppressor devices, the redirection and slowing of gasses which occurs from the baffling in a suppressor will cause increased back pressure in the gas impingement system. The increased back pressure will lead to increased bolt velocity (the rate at which the bolt is pushed back by the expanding gasses). With increased pressure, increased gas volume will also leave the bolt through the relief holes as seen in figure \ref{fig:Avg suppressor blow-back}. The unsuppressed case is shown in figure \ref{fig:unsuppressed fog} which has virtually no visible blow-back. The firearm shown is using what would be a typical competitor's suppressor system.}





\begin{figure}[h]
	\centering
	\includegraphics[scale=.3]{1.5 suppressor with wiper.png}
	\caption[Negative of Blow back]{This is a picture of blow-back on a morning that is not foggy when a 1.5 inch body suppressor is used. Unrelated to this data is the background which is the used microphone setup.}
	\label{fig:Avg suppressor blow-back}
\end{figure}



\subsection{Sound Reduction and Spectral Profile}

\paragraph{The standard of performance for a suppressor is currently sound reduction measured in decibels as a universal standard. The United States OSHA classifies an impulse sound reduced to below 140 decibels as hearing safe. Figure \ref{fig:Sound for Human Ear} we can see that the frequency of sound affects the pain threshold as well as the hearing threshold. This data motivates the study into the spectral density of suppressors and not solely the sound intensity. Equation \ref{eq:decible related to pressure} Shows the relation of sound pressure level and decibels.}



\begin{figure}
	\centering
	\includegraphics[scale=.45]{human ear.png}
	\caption[Human hearing diagram]{This figure depicts the human response to different frequencies and sound pressure levels. Regions of normal speech and music are shown. The threshold of hearing and pain are included. Note that those thresholds depend on frequency. \cite{commonswikimedia1}}
	\label{fig:Sound for Human Ear}
\end{figure}

\Large
\begin{equation}
	SPL=20\mathrm{dB}\log_{10} (\frac{P}{P_{ref}})
	\label{eq:decible related to pressure}
\end{equation}





\section{Applicability}

\paragraph{In the process of performing research and organized tests, Lonnie Jarvis was able to identify multiple variables which might reflect points for suppressor optimization. I made ways to measure these variables and made tables and graphs for ease to identify what did indeed create optimization. Because sound level is the conventional focus of suppressors this is a lead focus of Mission Silencers' suppressor and our goal to have sound pressure levels below 140 decibels at the operators ear. We collected data for bolt velocity and back pressure. This was in an effort to verify one company's claims that suppressors force too much pressure through gas tube and break AR style rifles and increase blow-back to unsafe levels. The claims were common with typical experiences of shooters who have used a suppressor on a semiautomatic rifle that is of an AR15 style. The conclusion by users of suppressed AR15 style rifles was it is not an enjoyable experience, either through breaking components or being gassed out from the blow-back in the operators face.} 

\paragraph{This research aimed to test Mission Silencers' suppressors with a patent pending new design which may change back pressure, bolt velocity and sound profile. By modifying the suppressor, we hoped to optimize four variables: decibels, blow-back, bolt velocity and spectral profile.}

\paragraph{Lastly, every suppressor has a different spectral profile regardless of sound pressure levels. The pitch produced by a suppressor seems to play a previously unrecognized role in the comfort to the operator's experience. Using a Fast Fourier Transform (FFT), the frequency content of the suppressed sound and their respective pressures were calculated and understanding was gained where there was previously only speculation. The method for finding the signal's spectrum mathematically is the Fourier transform by the integral in equation \ref{eq:Fast Fourier Transform}. This is a true Fourier transform, but with direction from Dr. Jon Johnson of Brigham Young University-Idaho we used a Fast Fourier Transform (FFT). A Fast Fourier transform (FFT) lets you see the components of sound and is a related technique of a Fourier Transform which is the below equation.}


\Large
\begin{equation}
	g(\omega)=\frac{1}{\sqrt{2 \pi}} \int_{-\infty}^{\infty}f(t)e^{i \omega t} dt
	\label{eq:Fast Fourier Transform}
\end{equation}